\documentclass{article}
\usepackage[utf8]{inputenc}
\usepackage[french]{babel}
\usepackage{amsthm}
\usepackage{amsmath}
\usepackage{amssymb}
\newtheorem*{theoreme}{Théorème}
\newtheorem*{lemme}{Lemme}
\newtheorem*{proposition}{Proposition}
\newtheorem{corollaire}{Corollaire}
\newtheorem*{remarque}{Remarque}
\theoremstyle{definition}
\newtheorem*{definition}{Définition}
\theoremstyle{remark}
\newtheorem*{demonstration}{Démonstration}
\usepackage[T1]{fontenc}
\setlength{\parindent}{0pt}
\usepackage{tikz}
\usepackage{float}
\usepackage[lowercase]{theoremref}
\DeclareMathAlphabet{\mathbbb}{U}{bbold}{m}{n}
\usepackage{geometry}
 \geometry{
 a4paper,
 total={170mm,257mm},
 left=20mm,
 top=20mm,
 }

\title{Sur le nombre de magmas de cardinal $n$}
\author{Matteo Wei}
\date{Juillet 2023}

\begin{document}

\maketitle

\section{La formule de Cauchy-Frobenius-Burnside}

\begin{definition}
    Soit $G$ un groupe de neutre $e$ et $X$ un ensemble. Une \textit{action de $G$ sur $X$} est une application $(g,x)\in G\times X \mapsto g.x\in X$ telle que :
    \begin{itemize}
        \item $\forall x\in X, e.x=x$.
        \item $\forall (g,h,x)\in G\times G\times X, g.(h.x)=(gh).x$.
    \end{itemize}
\end{definition}

On fixe pour la suite de cette partie $G$ un groupe fini, $X$ un ensemble fini, et $(g,x)\in G\times X \mapsto g.x\in X$ une action de $G$ sur $X$.

\begin{definition}
    On note :
    \begin{itemize}
        \item Pour $x\in X$, $\omega_G(x)=\{g.x\mid g\in G\}$ l'\textit{orbite} de $x$ sous l'action de $G$. Les orbites sous l'action de $G$ forment une partition de $X$.
        \item Pour $x\in X$, $G_x=\{g\in G\mid g.x=x\}$ le \textit{stabilisateur} de $x$ sous l'action de $G$. C'est un sous-groupe de $G$.
        \item Pour $g\in G$, $\mathrm{Fix}_X(g)=\{x\in X\mid g.x=x\}$ l'\textit{ensemble des points fixés} par $g$.
        \item $N(G,X)=|\{\omega_G(x)\mid x\in X\}|$ le nombre d'orbites de $X$ sous l'action de $G$.
    \end{itemize}
\end{definition}

\begin{definition}
    On dit que l'action de $G$ sur $X$ est \textit{libre} si tous les stabilisateurs sont triviaux.
\end{definition}

\begin{lemme}
    \thlabel{orb-stab}
    Soit $x\in X$. On a :
    $$|G|=|\omega_G(x)||G_x|.$$
\end{lemme}

\begin{demonstration}
    Si $(g,h)\in G^2$, on a $g.x=h.x\Leftrightarrow h^{-1}g.x=x\Leftrightarrow h^{-1}g\in G_x\Leftrightarrow g\in hG_x$, et par liberté de l'action de Cayley, $|hG_x|=|G_x|$. On en déduit facilement le résultat voulu.
\end{demonstration}

\begin{theoreme}[formule de Cauchy-Frobenius-Burnside]
    \thlabel{CFB}
    On a :
    $$N(G,X)=\frac{1}{|G|}\sum_{g\in G}|\mathrm{Fix}_X(g)|.$$
\end{theoreme}

\begin{demonstration}
    Soit $x\in X$. On remarque qu'on a :
    $$1=\sum_{y\in\omega_G(x)}\frac{1}{|\omega_G(x)|}.$$
    On en déduit que
    $$N(G,X)=\sum_{x\in X}\frac{1}{|\omega_G(x)|}.$$
    Il vient alors, d'après le \thref{orb-stab},
    \begin{align*}
        N(G,X)&=\sum_{x\in X}\frac{|G_x|}{|G|}\\
        &=\frac1{|G|}\sum_{x\in X}\sum_{g\in G}\mathbbb{1}_{g.x=x}\\
        &=\frac1{|G|}\sum_{g\in G}\sum_{x\in X}\mathbbb{1}_{g.x=x}\\
        &=\frac{1}{|G|}\sum_{g\in G}|\mathrm{Fix}_X(g)|
    \end{align*}
    ce qui conclut.
\end{demonstration}

\section{Dénombrement des magmas}

On fixe pour toute la suite $n\in\mathbb N$.

\begin{definition}
    Soit $\sigma\in\mathfrak S_n$. On note, pour $k\in[\![1,n]\!]$, $\mu(\sigma)_k$ le nombre de ses cycles de longueur $k$.
\end{definition}

\begin{proposition}
Il y a
$$\frac{1}{n!}\sum_{\sigma\in\mathfrak S_n}\prod_{1\leq k,l\leq n}\left(\sum_{d\mid k\lor l}d\mu(\sigma)_d\right)^{\mu(\sigma)_k\mu(\sigma)_lk\land l}$$
magmas de cardinal $n$ à isomorphisme près.
\end{proposition}

\begin{demonstration}
    On se ramène bien entendu sur l'ensemble $E=[\![1,n]\!]$ ; si $*$ et $\star$ sont deux lois de compositions internes sur $E$, on a par définition
    $$(E,*)\cong(E,\star)\Leftrightarrow\exists\sigma\in\mathfrak S_n:\forall (x,y)\in E^2,x*y=\sigma(\sigma^{-1}(x)\star\sigma^{-1}(y)).$$
    On pose donc naturellement, pour $\sigma\in\mathfrak S_n$ et $\star\in E^{E^2}$,
    $$\sigma.\star:(x,y)\in E^2\mapsto\sigma(\sigma^{-1}(x)\star\sigma^{-1}(y))\in E.$$
    On vérifie que $(\sigma,\star)\in \mathfrak S_n\times E^{E^2}\mapsto \sigma.\star\in E^{E^2}$ définit bien une action de $\mathfrak S_n$ sur $E^{E^2}$ ; ce qu'on vient de voir montre que deux lois définissent des magmas isomorphes si, et seulement si, elles sont dans la même orbite pour cette action. Ainsi, on cherche à déterminer $N\left(\mathfrak S_n,E^{E^2}\right)$, or la formule de Cauchy-Frobenius-Burnside nous dit que :
    $$N\left(\mathfrak S_n,E^{E^2}\right)=\frac{1}{n!}\sum_{\sigma\in \mathfrak S_n}\left|\mathrm{Fix}_{E^{E^2}}(\sigma)\right|.$$
    On est donc ramené au calcul des cardinaux des ensembles des lois fixées par l'action de $\mathcal S_n$.

    Soit $\sigma\in\mathfrak S_n$. Considérons une loi $\star$ fixée par $\sigma$. Cette condition se réécrit
    $$\forall(x,y)\in E^2, \sigma(x)\star\sigma(y)=\sigma(x\star y).$$
    On en déduit que :
    $$\forall k\in\mathbb N,\sigma^k(x)\star\sigma^k(y)=\sigma^k(x\star y).$$
    On est donc conduit à considérer les actions naturelles de $\langle\sigma\rangle$ sur $E^2$ et $E$. On fait deux observations :
    \begin{itemize}
        \item Choisir $x\star y$ détermine intégralement les valeurs de $\star$ sur $\omega_{\langle\sigma\rangle}(x,y)$.
        \item Pour que les choses bouclent correctement, il faut que $|\omega_{\langle\sigma\rangle}(x\star y)|\mid|\omega_{\langle\sigma\rangle}(x,y)|$.
    \end{itemize}
    
    Et on remarque facilement que ces conditions suffisent, c'est-à dire qu'en choisissant un élément vérifiant l'hypothèse de division des cardinaux pour chaque orbite pour l'action sur $E^2$, et en complétant correctement, on crée une loi fixée par $\sigma$.

    Soit $(k,l)\in[\![1,n]\!]^2$. Il y a par définition $\mu(\sigma)_k$ orbites de cardinal $k$ pour l'action sur $E$, et $\mu(\sigma)_l$ de cardinal $l$. Il y a donc $\mu(\sigma)_k\mu(\sigma)_lkl$ couples de la forme $(x,y)$, avec $|\omega_{\langle\sigma\rangle}(x)|=k$ et $|\omega_{\langle\sigma\rangle}(y)|=l$ ; ces couples se répartissent sur des orbites de cardinal $k\lor l$ pour l'action sur $E^2$, et il y a donc $\mu(\sigma)_k\mu(\sigma)_lk\land l$ telles orbites.

    Par ailleurs, d'après la deuxième condition proposée plus haut, on a pour chacune de ces orbites exactement
    $$\sum_{d\mid k\lor l}d\mu(\sigma)_d$$
    choix possibles d'un élément.

    On déduit de tout cela
    $$\left|\mathrm{Fix}_{E^{E^2}}(\sigma)\right|=\prod_{1\leq k,l\leq n}\left(\sum_{d\mid k\lor l}d\mu(\sigma)_d\right)^{\mu(\sigma)_k\mu(\sigma)_lk\land l}$$
    puis la formule voulue.
\end{demonstration}

On peut en déduire une formule dépendant uniquement des $n$-uplets de la forme des $\mu(\sigma)$, $\sigma\in\mathfrak S_n$ :

\begin{corollaire}
    \thlabel{cor1}
    Il y a
    $$\sum_{\substack{\mu\in \mathbb N^n\\\sum\limits_{k=1}^nk\mu_k=n}}\frac{\displaystyle\prod_{1\leq k,l\leq n}\left(\sum_{d\mid k\lor l}d\mu_d\right)^{\mu_k\mu_lk\land l}}{\displaystyle\prod_{k=1}^nk^{\mu_k}\mu_k!}$$
    magmas de cardinal $n$ à isomorphisme près.
\end{corollaire}

\begin{demonstration}
    Il s'agit de la formule précédente, combinée au dénombrement des permutations admettant un type cyclique donné.
    Pour construire une permutation admettant, pour $k\in[\![1,n]\!]$, $\mu_k$ $k$-cycles :
    
    \begin{itemize}
        \item On choisit quels éléments mettre dans chaque cycle, ce qui revient à un calcul d'anagrammes : il y a
        $\dfrac{n!}{\displaystyle\prod_{k=1}^nk!^{\mu_k}}$
        possibilités.
        \item On construit chacun des cycles à partir des éléments dedans : il y a $(k-1)!$ possibilités pour un cycle de longueur $k$, et donc $\displaystyle\prod_{k=1}^{n}(k-1)!^{\mu_k}$ possibilités au total.
        \item On constate qu'on construit une permutation donnée autant de fois qu'on peut permuter les cycles de même longueur entre eux, il faut donc diviser par $\displaystyle\prod_{k=1}^n\mu_k!$.
    \end{itemize}

    Cela donne, au total,
    $$\frac{n!}{\displaystyle\prod_{k=1}^nk^{\mu_k}\mu_k!}$$
    telles permutations, ce qui est bien le résultat voulu.
\end{demonstration}

On en déduit aussi des expressions dépendant des types cycliques.

\begin{definition}
    Soit $\sigma\in\mathfrak S_n$. On note $\lambda(\sigma)=(\lambda(\sigma)_k)_{k=0}^{r(\sigma)}$ son type cyclique.
\end{definition}

\begin{corollaire}
    \thlabel{cor2}
    Il y a
    $$\frac{1}{n!}\sum_{\sigma\in\mathfrak S_n}\prod_{1\leq k,l\leq r(\sigma)}\left(\sum_{\substack{1\leq d\leq r(\sigma)\\\lambda(\sigma)_d\mid \lambda(\sigma)_k\lor \lambda(\sigma)_l}}\lambda(\sigma)_d\right)^{\lambda(\sigma)_k\land \lambda(\sigma)_l}$$
magmas de cardinal $n$ à isomorphisme près.
\end{corollaire}

\begin{demonstration}
    Soit $\sigma\in\mathfrak S_n$. Remarquons que, si $(k,l)\in [\![1,n]\!]^2$, on a :
    $$\sum_{\substack{1\leq d\leq r(\sigma)\\\lambda(\sigma)_d\mid k\lor l}}\lambda(\sigma)_d=\sum_{d\mid k\lor l}d\mu(\sigma)_d.$$
    En effet, ces deux quantités sont le nombre d'éléments dans des cycles de longueurs divisant $k\lor l$.
    Par ailleurs, comme il y a $\mu(\sigma)_k$ cycles de longueur $k$, et $\mu(\sigma)_l$ de longueur $l$,
    $$\left(\sum_{\substack{1\leq d\leq r(\sigma)\\\lambda(\sigma)_d\mid k\lor l}}\lambda(\sigma)_d\right)^{k\land l}$$
    apparaît $\mu(\sigma)_k\mu(\sigma)_l$ fois dans le produit du \thref{cor2}, ce qui donne bien le résultat voulu.
\end{demonstration}

L'expression correspondante à celle du \thref{cor1}, en fonction des partitions de $n$, est moins intéressante car elle nécessite quand même de compter les cycles d'une longueur donnée.

\end{document}
